\raggedright
\footnotesize
\begin{multicols}{3}	
	\setlength{\premulticols}{1pt}
	\setlength{\postmulticols}{1pt}
	\setlength{\multicolsep}{1pt}
	\setlength{\columnsep}{2pt}

\begin{center}
     \Large{\underline{Programmiersprachen}} \\
\end{center}

\section{Arten von Programmiersprachen}
\subsection{General purpose languages}
Mit GPL-Programmiersprachen können eine Vielzahl von Programmen entwickelt werden. Beispielsweise kann man mit GPL-Programmiersprachen oft sowohl Konsolenprogramme als auch Programme mit einer grafischen Benutzeroberfläche programmieren. GPL-Programmiersprachen beinhalten darüber hinaus typischerweise oft Elemente wie If,/else-, for-, while-Statements und erlauben das Programmieren und den Aufruf von \enquote{Unterfunktionen}. Ferner sind GPL-Programmiersprachen in aller Regel turing-vollständig.
\subsection{Domain specific languages}
DSL-Programmiersprachen sind nicht dafür geeignet, beliebige Softwareentwicklungs-projekte zu realisieren. Die Anwendbarkeit von DSL-Programmiersprachen sind auf (oft sehr wenige) \enquote{Domänen} beschränkt. Beispiel: SQL. SQL-Statements sind dafür geeignet, um Daten aus einer Datenbank abzufragen, hinzuzufügen oder zu editieren. In dieser \enquote{Domäne} hat SQL allen anderen Programmiersprachen einen wesentlichen Vorteil. Im Gegensatz zu GPL-Programmiersprachen ist SQL (wie auch alle anderen DSL-Programmiersprachen) allerdings ungeeignet, um etwa ein ausführbares Programm zu entwickeln.
\subsection{Deklarative Sprachen}

\section{Ausführung des Programmcodes zur Laufzeit}
\subsection{Assembler}
Ein Assemblierungsprogramm übersetzt Assembler-Befehle direkt in Maschinencode. Beim Umkehrvorgang (Disassemblieren) kann der Assemblercode weitesgehend wieder hergestellt werden. Direkt in Assembler zu programmieren ist heute nur noch selten gebräuchlich und nicht praxistauglich für große Softwareprojekte. Für spezielle Anwendungen kann es allerdings manchmal sinnvoll (oder sogar erforderlich) sein, Assemblercode zu schreiben. Anbieten tut sich dies bei speziellen Aufgaben, die hinsichtlich der Performance optimiert werden sollen. Da Programmierung in Assembler die hardwarenächste Art ist, zu programmieren, die in der Praxis auffindbar ist. 
\subsection{Kompilierung}
Bei Sprachen wie C oder C++ wird der Programmcode kompiliert. Das bedeutet, dass der Programmcode vom Compiler in Maschinencode-Befehle übersetzt wird, wodurch der Programmierer durch den Kompilier-Vorgang eine ausführbare Datei (z. B. .exe-Datei) erhält. Dies kann mit Aufrufen wie
\begin{minted}
gcc MyProgram.c -o MyProgram.exe 
\end{minted}
erfolgen. Programme von den meisten GPL-Programmiersprachen müssen vor der Ausführung kompiliert werden. Der Umkehrvorgang zur Kompilierung ist Dekompilierung. Es gibt Tools, zur Dekompilierung von Programmen, allerdings ist die Dekompilierung nicht eindeutig und bringt daher als Ergebnis nicht den Original-Programmcode eines Programms hervor. Dies kann im Wesentlichen folgende Gründe haben:\\
\begin{itemize}
\item Der Compiler optimiert den Programmcode in der Regel soweit wie möglich, um ihn beispielsweise performanter zu machen.
\item Es gibt mehrere verschiedene Programmcode-\enquote{Varianten} für verschiedene (wenn auch sehr ähnliche) Programmcode-Statements, die zum gleichen Maschinencode führen. Der Decompiler kann nur schwer feststellen, welche dieser Varianten im Original-Programmcode verwendet worden ist.
\item Der Programmcode wurde obfuskiert. Hier werden beim Compilieren (oder ggf. auch davor oder danach) technische Maßnahmen angewandt, um es Dekompilierern und Reverse-Engineerern möglichst schwer zu machen, den Programmcode wieder herzustellen.
\end{itemize}
\subsection{Interpretation}
Bei der Interpretation wird ein Programm nicht kompiliert, sondern der Programmcode selbst wird dem Anwender ausgeliefert. Programmiersprachen, deren Name auf \enquote{Script} endet, sind in aller Regel Programmiersprachen, die interpretiert werden. Der Anwender braucht zum Ausführen des Programms einen Interpreter. Dieser \enquote{Interpretationvorgang}, der zur Ausführung des Programms führt, dauert länger als die Ausführung eines kompilierten Programms, da z. B. erst der Programmcode geparst werden muss. Programme von DSL-Programmiersprachen werden oft interpretiert statt kompiliert. Browser haben daher z. B. immer einen Javascript-Interpreter und Datenbanken haben z. B.  immer einen SQL-Interpreter. Dadurch das der Programmcode selbst zur Ausführung gebraucht wird, sind Programme einer DSL immer \enquote{open source}\footnote{Dies heißt natürlich nicht, dass Programme von DSL-Programmiersprachen immer Lizenzkostenfrei sind. Die Lizenz eines Programms einer DSL-Programmiersprache ist zu prüfen unabhängig davon, ob man dessen Programmcode besitzt oder nicht.}, bzw. genauer gesagt: Ein Anwender des Programms braucht (neben dem Interpreter) den Programmcode und kann ihn daher im Gegensatzu zu kompiliertem Code beispielsweise theoretisch auf Fehler/Schadcode überprüfen. Dies wird jedoch dadurch erschwert, dass Programmcode von DSL-Programmiersprachen oft \enquote{uglified} und \enquote{minified} an den Anwender herausgegeben wird, um Speicherplatz bei der Übertragung zu sparen\footnote{Diese Optimierung wird speziell bei Javascript sehr extrem angewandt, um die Anzahl an Daten zu verkleinern, die ein Internetsurfer beim Aufruf einer Seite downloaden muss, ohne deren Semantik zu verändern.}.
\subsubsection{Skripte}
Als Skript bezeichnet man eine Datei, welche beliebig viele (von Menschen lesbare) Befehle enthält, die bei Ausführung von einem geeigneten Interpreter interpretiert werden. Gängige Beispiele sind:\\
\begin{itemize}
\item Javascript-Dateien
\item SQL-Dateien
\item Kommandozeileninterpreter-Skripte (z. B. .bat-Dateien, .sh-Dateien; Diese Dateien brauchen keinen externen Interpreter, da der Interpreter direkt im Betriebssystem verfügbar ist)
\end{itemize}
Skriptdateien sind nicht zu verwechseln mit Konfigurationsdateien.
\subsection{Zwischensprache}
Einige Programmiersprachen (im Wesentlichen C\#, VB.Net und Java) kompilieren den Code zwar scheinbar zu einer Ausführbaren Datei, jedoch beinhaltet diese ausführbare Datei nicht direkt Maschinencode, sondern einen (für Menschen nicht lesbaren) Zwischencode (sogenannten Bytecode), der von einer installierten Laufzeitumgebung interpretiert und ausgeführt wird. Dieser Zwischencode ist stark optimiert um die Ausführungsgeschwindigkeit zu erhöhren, trotzdem sind Sprachen, deren Programme in eine Zwischensprache kompiliert wird\footnote{Das Wort \enquote{kompilieren} wird im Allgemeinen auch für den Vorgang benutzt, in dem aus dem Programme von Sprachen wie C\#, Java usw. in scheinbar ausführbare Dateien erzeugt werden.}, langsamer als Sprachen, die direkt zu richtigem Maschinencode kompiliert werden.\\
Um .Net-Programme (Programme, die mit C\# (oder anderen .Net-Programmiersprachen von Microsoft) geschrieben worden sind) auszuführen, wird das .Net-Framework benötigt, welches quasi nur auf Windows anwendbar ist\footnote{Es gibt mit \enquote{.Net Code} und \enquote{.Net Standard} Ansätze von Microsoft, um .Net-Programme auch auf anderen Plattformen lauffähig zu machen}. Dies kann kostenfrei gedownloaded/benutzt werden und ist bei Windows 10 beispielsweise vorinstalliert.\\
Um Java-Programme auszuführen, wird eine installierte Java-Runtime-Environment benötigt. Diese ist für viele gängige Plattformen kostenfrei verfügbar.

\section{Generationen von Programmiersprachen}
\subsection{Programmiersprachen erster Generation}
Programmiersprachen erster generation sind quasi keine richtigen Programmiersprachen. Bevor die späteren Generationen von Programmiersprachen entwickelt worden sind, mussten Programmierer den Maschinencode selbst schreiben. Es gab keinen Compiler oder ähnliches. Die Programmierarbeit war äußerst aufwendig. Diese Verfahren sind heute in der Praxis nicht mehr gebräuchlich.
\subsection{Programmiersprachen zweiter Generation}
Programmiersprachen dritter Generation bezeichnen üblicherweise Assembler-Programmiersprachen. Sie zeichnen sich dadurch aus, dass es ein Assemblierungsprogramm gibt, der die definierten für Menschen lesbaren Befehle in Maschinencode übersetzt.
\subsection{Programmiersprachen dritter Generation}
Programmiersprachen dritter Generation sind Sprachen wie C, Fortran, COBOL, Pascal, C\#, C++. Sie wurden entwickelt, um dem Programmierer die mühselige Arbeit zu entwickeln, Assemblercode zu schreiben. Sie zeichnen sich dadurch aus, dass es einen Compiler gibt, der aus dem Programmcode Maschinencode oder Code einer Zwischensprache erzeugt.
\section{Gängige Programmiersprachen}
\subsection{C}
C ist relativ hardwarenah und sehr weit verbreitet. C bietet sich für Softwareprojekte an, die auf mehreren Plattformen ausgeführt werden sollen, da es für quasi jede heute gebräuchliche Plattform einen C-Compiler gibt. Es ist möglich, Assemblerprogramme in C zu integrieren, um z. B. bestimmte Aufgaben, bei denen Performance wichtig ist, direkt in Assembler programmieren zu können.
\subsection{C++}
C++ ist eine Erweiterung der Sprachspezifikation von C. Jedes C-Programm ist daher mit einem C++-Compiler kompilierbar. C++-Programme, sind, wenn sie C++-spezifische Features benutzen, umgekehrt nicht mit einem C-Compiler kompilierbar. C++ bietet diverse Features wie etwas Klassen/Mehrfachvererbung und ist eine relativ häufig eingesetzte Programmiersprache.
\subsection{Javascript}
Javascript (nicht zu verwechseln mit Java) wird überwiegend auf Websites eingesetzt, um diese interaktiv zu gestalten.
\subsection{C\#}
C\# ist eine verbreitete Programmiersprache von Microsoft, deren Syntax an C angelehnt ist und es Entwicklern relativ einfach machen soll, schnell auch komplexe Anwendungen zu entwickeln. C\#-Programme werden beim kompilieren in Bytecode der Common Intermediate Language übersetzt und benötigen das .Net-Framework zur Ausführung. C\# erlaubt die Verwendung von Bibliotheken, die in VB.Net geschrieben worden sind.
\subsection{VB.Net}
VB.Net (auch \enquote{Visual Basic .Net}, nicht zu verwechseln mit VB\footnote{Umgangssprachlich wird oft \enquote{VB} bzw. \enquote{Visual Basic} gesagt, wenn in Wirklichkeit \enquote{VB.Net} bzw. \enquote{Visual Basic .Net} gemeint ist.}) ist eine programmiersprache, die de facto die gleiche Mächtigkeit und den gleichen Zweck hat wie C\#, syntaktisch aber auf VB basiert. Darüber hinaus ist VB.Net genau wie C\# auch von Microsoft entwickelt. VB.Net-Programme werden beim kompilieren in Bytecode der Common Intermediate Language übersetzt und benötigen das .Net-Framework zur Ausführung. Der Fokus bei VB.Net liegt auf eine Syntax, die an die menschliche Sprache angelegt ist, um es Programmieranfängern zu ermöglichen, Programmcode einer objektorientierten Programmiersprache einfacher lesen zu können. Programmcode von VB.Net und C\# ist bis auf wenige Ausnahmen ineinander konvertierbar. Einige Ausnahmen bzw. Unterschiede zu C\#:\\
\begin{itemize}
\item In VB.Net sind readonly-properties parametrisierbar, bei C\# nicht.
\item In VB.Net sind Funktionsnamen case-insensitive. (In C\# sind sie case-sensitive.)
\end{itemize}
Langfristig plant Microsoft, die Feature-Entwicklung auf C\# zu fokussieren und weniger Wert auf weitesgehende Konvertierbarkeit von C\#- und VB.Net-Programmcode zu legen, sodass die Anzahl der Features, die C\# hat und nicht in VB.Net nutzbar sind, in Zukunft steigen dürfte. Vb.Net erlaubt die Verwendung von Bibliotheken, die in C\# geschrieben worden sind (eventuell treten Probleme wegen unterschiedlicher case-sensitivity auf).
\subsection{VB}
VB (auch \enquote{Visual Basic}, nicht zu verwechseln mit VB.Net) ist eine Skriptsprache und der Vorläufer von VB.Net. Syntaktisch liegt der Fokus auf eine einfache Lesbarkeit. VB-Skripte werden heute noch oft in Excel-Dokumenten verwendet.
\subsection{Java}
Java (nicht zu verwechseln mit Javascript) 
\subsection{Typescript}
Typescript ist eine Erweiterung von Javascript, die Elemente aus der objektorientierten Programmierung beinhaltet. Es soll damit für Entwickler möglich sein, die zunehmend komplexer werdende Dynamik von Websites mit Unterstützung von Elementen wie Klassen und Interfaces zu entwickeln, die es so in Javascript nicht direkt gibt. Typescript-Code wird bei Produktivsetzung in Javascript übersetzt (\enquote{transkompiliert}).
\subsection{Coffeescript}
Coffeescript ist (ähnlich wie Typescript) eine andere Programmiersprache um Javascript-Code zu schreiben. Coffeescript legt den Fokus auf bessere Lesbarkeit und Vermeidung von Klammern. Coffeescript-Code wird bei Produktivsetzung in Javascript übersetzt (\enquote{transkompiliert}).
\subsection{Lua}
Ist eine imperative Skript-Sprache, die in vielen Arten von Programmen (auch oft in Spielen) zum Einsatz kommt, um Dinge einfach konfigurierbar/skriptbar zu machen.
\subsection{SQL}
SQL ist eine Abfragefrage für Datenbanken. SQL gehört zu den domänenspezifischen Sprachen.
\subsection{XPath}
XPath ist eine Abfragesprache um Daten aus einem XML-Dokument auszulesen. XPath erfüllt analog anschaulich gesagt den gleichen Zweck wie SQL für Datenbanken. XPath gehört zu den domänenspezifischen Sprachen.
\subsection{XQuery}
XQuery ist eine Abfragesprache um Daten aus einem XML-Dokument auszulesen. XQuery ist mächtiger als XPath, da es wesentlich mehr Funktionen zur Verfügung hat. XQuery gehört zu den domänenspezifischen Sprachen. Ferner ist XQuery ferner turing-vollständig.
\subsection{Python}
Python ist eine Skriptsprache, die es ermöglicht, komplexe längere Programme zu schreiben, die vom Funktionsumfang her GPL-Programmiersprachen nahekommt. Python ist plattform-unabhängig und wird benutzt um diverse Sachen einafch und schnell skripten zu können.
\subsection{PHP}
PHP ist eine Skriptsprache, die vor z. B. oft bei Webservern eingesetzt wird, um den Inhalt einer Website beim Abruf dynamisch zu erzeugen (durch PHP-Skripte berechnen zu lassen).
\subsection{Go}
Go ist eine relativ neue urpsrünglich von Google entwickelte Programmiersprache, die kompiliert wird und den Fokus darauf legt, Nachteile von C++ oder Java etwa bei hinsichtlich der Entwicklung für skallierbare Netzwerkdienste zu vermeiden.
\subsection{Smalltalk}
Smalltalk ist eine sehr alte objektorientierte Programmiersprache. In Smalltalk ist \enquote{alles} ein Objekt, also auch Zeiger und \enquote{primitive Daten} wie etwa Integer- und Boolean-Werte. Smalltalk wird in Bytecode kompiliert.
\subsection{Objective-C}
Objective-C ist eine Erweiterung von C. Objective-C-Kompiler können daher auch C-Programme kompilieren.
\subsection{Swift}
Swift ist eine von Apple entwickelte Programmiersprache, die primär für die Ausführung von Programmen auf Apple-Geräte konzipiert worden ist.
\subsection{R}
R ist eine domänenspezifische Programmiersprache, die für statistische Berechnungen konzipiert worden ist.
\end{multicols}

