\raggedright
\footnotesize
\begin{multicols}{3}	
	\setlength{\premulticols}{1pt}
	\setlength{\postmulticols}{1pt}
	\setlength{\multicolsep}{1pt}
	\setlength{\columnsep}{2pt}

\begin{center}
     \Large{\underline{Assembler}} \\
\end{center}

\section{Allgemeines}\footnote{Dieses Cheatsheet bezieht sich hauptsächlich auf IA-32-Assembler}

\settowidth{\MyLen}{\texttt{option.2.spa}}
Als Assembler bezeichnet man Computerprogramme, die Assemblerbefehle in Maschinencode übersetzt. Im Gegensatz zu Compilern von Hochsprachen übersetzen Assembler strikt die eingegebenen Befehle und interpretieren den den Eingangsquellcode kaum.
\section{Register}
\subsection{Verwendung der Register}
General purpose Register:
\begin{itemize}
\item eax: Zwischenwerte/Rückgabewerte bei Berechnungen
\item ebx: Adressierungen (Base)
\item ecx: Zählerregister (Counter)
\item edx: I/O-Daten (Data)
\item esi: Quelloperand-Speicheradresse für Stringoperationen (Source)
\item edi: Zieloperand-Speicheradresse für Stringoperationen (Destination)
\end{itemize}
Special purpose Register:
\begin{itemize}
\item esp: Enthält die Adresse des obersten Stackelements (Stackpointer)
\item ebp: Enthält die Adresse des aktuellen Stack-Frames
\item eip: Enthält die aktuell auszuführende Instruktion (Instructionpointer)
\item eflags: Enthält diverse Flags (Zeroflag, Overflow-Flag usw.)
\end{itemize}
Segment-Register:
\begin{itemize}
\item cs: Codesegment
\item ds: Datasegment
\item es: Extrasegment
\item ss: Stacksegment
\end{itemize}
\subsection{Verwendung der Flags}
Die folgende Auflistung enthält die Flags, die im Flag-Register gespeichert sind.
\begin{itemize}
\item CF (Carry-Flag): Enthält den Übertrag aus einer vorangegangenen Operation
\item PF (Parity-Flag): TODO
\item AF (Adjust-Flag): TODO
\item ZF (Zero-Flag): Ist 1, wenn das Ergebnis der letzten Operation 0 war.
\item SF (Sign-Flag): TODO
\item TF (Trap-Flag): TODO
\item IF (Interrupt-Enabled-Flag): TODO
\item DF (Direction-Flag): TODO
\item OF (Overflow-Flag): Gibt an ob bei der letzten Operation ein Überlauf (oder \enquote{Unterlauf}) aufgetreten ist. Gewöhnlich definiert als OF=in-carry\footnote{Bezeichnet das Übertragsbit, das in die Vorzeichenstelle hineingeht} xor out-carry\footnote{Bezeichnet das Übertragsbit, das aus der Vorzeichenstelle hinausgeht}
\item IOPL (IO-Privilege-Level): TODO
\item NT (Nested-Task): TODO
\item RF (Resume-Flag): TODO
\item VM (Virtuel-8086-Mode): TODO
\item AC (Alignment-Check): TODO
\item VIF (Virtual-Interrupt-Flag): TODO
\item VIP (Virtual-Interrupt-Pending): TODO
\item ID (Able to use CPUID instruction): TODO
\end{itemize}
\section{Adressierungsarten}
\section{Befehle}
\section{Common Intermediate Language}
\subsection{mov}
\subsection{sub}
\subsection{call}

\end{multicols}