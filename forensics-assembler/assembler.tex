\raggedright
\footnotesize
\begin{multicols}{3}	
	\setlength{\premulticols}{1pt}
	\setlength{\postmulticols}{1pt}
	\setlength{\multicolsep}{1pt}
	\setlength{\columnsep}{2pt}

\begin{center}
     \Large{\underline{Assembler (allgemein)}} \\
\end{center}

\section{Allgemeines}\footnote{Dieses Cheatsheet bezieht sich auf IA-32-Assembler}
\settowidth{\MyLen}{\texttt{option.2.spa}}
\section{Register}
Verwendung von Registern:
\begin{itemize}
\item eax: Zwischenwerte/Rückgabewerte bei Berechnungen
\item ebx: Adressierungen (Base)
\item ecx: Zählerregister (Counter)
\item edx: I/O-Daten (Data)
\item esi: Quelloperand-Speicheradresse für Stringoperationen (Source)
\item edi: Zieloperand-Speicheradresse für Stringoperationen (Destination)
\item esp: Enthält die Adresse des obersten Stackelements (Stackpointer)
\item ebp: Enthält die Adresse des aktuellen Stack-Frames
\item eip: Enthält die aktuell auszuführende Instruktion (Instructionpointer)
\item eflags: Enthält diverse Flags (Zeroflag, Overflow-Flag usw.)
\end{itemize}
\end{multicols}