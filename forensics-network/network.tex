\raggedright
\footnotesize
\begin{multicols}{3}	
	\setlength{\premulticols}{1pt}
	\setlength{\postmulticols}{1pt}
	\setlength{\multicolsep}{1pt}
	\setlength{\columnsep}{2pt}

\begin{center}
     \Large{\underline{Netzwerkforensik (allgemein)}} \\
\end{center}

\section{Sniffing}
\section{Tools}
\settowidth{\MyLen}{\texttt{option.2.spa}}
\begin{tabular}{@{}p{\the\MyLen}
		@{}p{\linewidth-\the\MyLen}@{}}
	\texttt{cURL} & Einfaches Programm zum Senden von Netzwerk-Requests. Unterstützte Protokolle sind unter anderem HTTP, HTTPS, FTP und FTPS.\\
	\texttt{dig} & TODO\\
	\texttt{dsniff} & TODO\\
	\texttt{Ettercap} & Tool zum Durchführen von Man-in-the-middle-Angriffen, beispielsweise mittels ARP-Spoofing.\\
	\texttt{filesnarf} & TODO\\
	\texttt{mailsnarf} & TODO\\
	\texttt{msgsnarf} & Sniffer für ältere bekannte Chat-Messenger (ICQ, IRC, MSN Messenger usw.)\\
	\texttt{nmap} & Etablierter Konsolen-basierter Portscanner.\\
	\texttt{Scapy} & Tool zum Manipulieren von Paketen im Netzwerkverkehr.\\
	\texttt{urlsnarf} & Sniffer für HTTP-Requests\\
	\texttt{pcap} & TODO\\
	\texttt{Tcpdump} & TODO\\
	\texttt{webspy} & TODO\\
	\texttt{Wireshark} & Etablierter Netzwerksniffer für Pakete verschiedener Protokolle\\
\end{tabular}
\lipsum
\end{multicols}