%Preamble for every 
\raggedright
\footnotesize
\begin{multicols}{3}	
	% multicol parameters
	% These lengths are set only within the two main columns
	%\setlength{\columnseprule}{0.25pt}
	\setlength{\premulticols}{1pt}
	\setlength{\postmulticols}{1pt}
	\setlength{\multicolsep}{1pt}
	\setlength{\columnsep}{2pt}

\begin{center}
     \Large{\underline{Windows 10-Forensik}} \\
\end{center}

\section{Allgemein}

\subsubsection{Buildnummer}
Aktuelle Buildnummer über \texttt{systeminfo} (cmd.exe) oder \path{HKLM\Software\Microsoft\Windows NT\CurrentVersion\CurrentBuildNumber}

\subsubsection{Zuletzt verwendete Elemente}
\path{C:\Users\<username>\AppData\Roaming\Microsoft\Windows\Recent}

\section{Überwachter Ordnerzugriff}
Überwacht und blockiert den schreibenden Zugriff auf vorhandene Dateien für nicht-vertrauenswürdige Applikationen.


\subsubsection{Aktivieren}
Windows Defender Security Center $\rightarrow$ Einstellungen für Viren- und Bedrohungsschutz $\rightarrow$ Überwachter Ordnerzugriff\\
\texttt{oder}\\
Gruppenrichtlinien: \texttt{Computerkonfiguration/Administrative Vorlagen/Windows/Windows Defender Antivir/Windows Defender Exploit Guard/Überwachter Ordnerzugriff}\\
\texttt{oder}\\
Registry (Besitzer vorher ändern): \path{HKLM\Software\Microsoft\Windows Defender\Windows Defender Exploit Guard\ControlledFolderAccess\EnableControlledFolderAccess} (DWORD) = 0x01

\subsubsection{Erlaubte Anwendungen}
\path{HKLM\Software\Microsoft\Windows Defender\Windows Defender Exploit Guard\ControlledFolderAccess\AllowedApplications}\\
Hinzufügen mit (PS): \texttt{Add-MpPreference -ControlledFolderAcessAllowedApplications "<Anwendungspfad>"}

\subsubsection{Geschützte Ordner}
\path{HKLM\Software\Microsoft\Windows Defender\Windows Defender Exploit Guard\ControlledFolderAccess\ProtectedFolders}\\
Standardmäßig geschützte Ordner: Documents|Pictures|Videos|Music|Desktop|Favorites (<username> und Public)\\


\subsubsection{Ereignisse}
Einzusehen über EventVwr
oder Powershell:\\
\texttt{Get-WinEvent -LogName "Microsoft-Windows-Windows Defender/Operational" | Where-Object \{\$\_.Id -in 1123,1124,5007\}}\\
Ereignis-IDs:
\settowidth{\MyLen}{\texttt{5007.spa}}
\begin{tabular}{@{}p{\the\MyLen}%
		@{}p{\linewidth-\the\MyLen}@{}}
	\texttt{1123} & Blockiertes Ereignis\\
	\texttt{1124} & Überwachtes Ereignis (Auditmodus)\\
	\texttt{5007} & Änderung von Einstellungen\\
\end{tabular}


\section{Jumplists}
Mehr Informationen als MRU/MFU:
\begin{itemize}[leftmargin=*]
	\item Dateiname, -pfad
	\item MAC Zeitstempel
	\item Name des Volumes
	\item Zeitlicher Verlauf von Down- und Uploads
	\item Informationen bleiben nach Löschen der Datei erhalten
\end{itemize}

\subsubsection{Speicherort}
Erstellt vom Betriebssystem: \path{C:\User\<username>\AppData\Roaming\Microsoft\Windows\Recent\AutomaticDestinations}\\
Erstellt von Softwareanwendungen: \path{C:\User\<username>\AppData\Roaming\Microsoft\Windows\Recent\CustomDestinations}\\
Dateiname: \path{<AppId>.<automatic|custom>Destinations-ms}\\
Die AppId kann im ForensicsWiki nachgelesen werden \url{https://www.forensicswiki.org/wiki/List_of_Jump_List_IDs}

\subsubsection{AutomaticDestination JL}
Aufbau der Datei:\\
Header (32 Byte) mit Versionssnummer (3=Win10, 1=Win7/8), Anzahl Einträge, Anzahl gepinnte Einträge, Zuletzt zugewiesene Entry-ID, Anzahl der Aktionen\\
DestList-Entry:\\
\settowidth{\MyLen}{\texttt{volume.id.space.l}}
\begin{tabular}{@{}p{\the\MyLen}%
		@{}p{\linewidth-\the\MyLen}@{}}
	\texttt{Prüfsumme} &  Fehlerhafter Eintrag wird nicht angezeigt\\
	\texttt{(New|Birth) Volume-ID} &  Bei Änderung des Volumes geänderte New-ID\\
	\texttt{(New|Birth) Object-ID} &  Generiert aus Bootzeit,Sequenznummer und MAC-Adresse. Bei Änderung des Volumes neue New-ID\\
	\texttt{NetBios Name} & nbtstat -n\\
	\texttt{Entry ID} & Fortlaufende Nummer\\
	\texttt{Access Timestamp} & letzter Zugriff \\
	\texttt{Pinned Status} & angepinnt (ja/nein)\\
	\texttt{Access Count} & Zugriffszähler\\
	\texttt{variabel Unicode} & vollständiger Pfad zur Datei\\
	\texttt{Länge Unicode} & Länge Unicodepfad\\
\end{tabular}

\subsubsection{CustomDestionations JL}
einfachere Dateistruktur, zusammengesetzte MS-SHLINK-Segmente\\
Anfang eines LNK-Segments: \texttt{4C 00 00 00 01 14 02 00 00 00 00 00 C0 00 00 00 00 00 00 46}\\
Ende: \texttt{AB FB BF BA}

\subsubsection{QuickAccess/Schnellzugriff}
Angepinnte Einträge im Schnellzugriff des Explorer. Dateiname \path{5f7b5f1e01b83767.automaticDestinations-ms}

\subsubsection{Tools}
\settowidth{\MyLen}{\texttt{volume.id.space.l}}
\begin{tabular}{@{}p{\the\MyLen}%
		@{}p{\linewidth-\the\MyLen}@{}}
	\texttt{JumpListExt for Windows 10} & grafische Oberfläche, nicht mehr stabil in aktuellen Versionen\\
	\texttt{JLECmd} & \texttt{JLECmd.exe -f <JLFile> (--html|--csv|--json) <targetDir> (--ld)}\\
\end{tabular}

\section{Windows 10 Applications}
\subsubsection{SystemApps}
vorinstalliert, können nicht deinstalliert werden\\
\path{C:\Windows\SystemApps\<appname>}
\subsubsection{WindowsApps}
über Windows Store
\path{C:\Windows\WindowsApps\<appname>}
\subsubsection{Einstellungsdaten}
\path{C:\Users\<username>\AppData\Local\Packages\<appname>}\\
Haupteinstellungen in Datei/Registry-Hive \path{settings.dat}
\subsubsection{Anwendungsdaten}
Gespeichert in ESE-DB-Datenbanken, Aufbau nicht vollständig bekannt, teilweise möglich mit \texttt{ESEDatabaseView} von Nirsoft

\section{Build-in applications}
Im Folgenden sind auf Windows bereits vorinstallierte Programme aufgelistet, die forensisch verwertbare Information bringen können, mit dem Namen, unter dem sie im Konsolen-/Powershell-/\enquote{Ausführen}-/\enquote{Neuen Task ausführen}-Fenster gestartet werden können:

\begin{tabular}{@{}p{\the\MyLen}
		@{}p{\linewidth-\the\MyLen}@{}}
	\texttt{certmgr} &  Tool zum Verwalten der für den jeweiligen Benutzer verfügbaren Zertifikate.\\
    \texttt{cipher} &  Tool zum sicheren löschen von Datan, sodass sie nicht wieder herstellbar sind. Kann auch dafür verwendet werden, freien Speicherplatz auf der Festplatte zu löschen. Kann auch dafür verwendet werden, Dateien zu verschlüsseln.\\
	\texttt{diskmgmt} &  Tool mit grafischer Oberfläche zum Verwalten von Datenträgern: Partitionen, Laufwerksbuchstaben und die Partitionstabellenart (MBR/GPT) von Datenträgern kann hiermit verändert werden\\
	\texttt{diskpart} &  Kommandozeilentool, das ähnliche Funktionalität bietet wie diskmgmt.\\
	\texttt{eventvwr} & Tool zum Anzeigen diverser systemweiter Ereignisse. Entwickler von Dritt-Programmen können ihre Programme ebenfalls Ereignisse in die Ereignisanzeige schreiben lassen.\\
	\texttt{gpedit} & Editor zum Bearbeiten von Richtlinien für einzelne Benutzer oder den ganzen Computer. Hier können Sicherheitseinstellungen vorgenommen werden aber auch Skripte hinterlegt werden, die beim Anmelden/Abmelden eines Nutzers oder auch beim Starten/Herunterfahren des Computers ausgeführt werden.\\
	\texttt{msconfig} & Bietet Konfigurationsmöglichkeiten für den Start des Systems und bietet darüber hinaus eine Anzeige zur Information, welche Dienste gerade ausgeführt werden und welche davon beim Systemstart gestartet werden.\\
	\texttt{msinfo32} & Liefert ausführliche Informationen zu Treibern, angeschlossene Hardware, Druckaufträge, Systemvariablen, geladene Module, Dienste, etc.\\
	\texttt{perfmon} & Systemleistungs-Monitoring-Tool. Kann dazu benutzt werden, Statistiken über einzelne Prozesse und Eigenschaften einzelner Prozesse aufzuzeichnen.\\
	\texttt{regedit} & Editor für die Registry.\\
	\texttt{resmon} & Tool zum Monitoring von CPU, RAM, Prozessen, Netzwerkschnittstellen und Datenträgern.\\
	\texttt{secpol} & Editor zum Einstellen diverser Richtlinien. Es kann z. B. eingestellt werden, welche Ereignisse überwacht oder sogar unterbunden werden sollen.\\
	\texttt{taskschd} & Tool zum Anlegen von Aufgaben, die regelmäßig bzw. unter bestimmten Bedingungen ausgeführt werden.\\
	\texttt{WF} &  Bietet Firewall-Konfigurationsmöglichkeiten
	\end{tabular}

\section{Scripts}
Sicherstellen, dass eine Batch-Datei als Administrator gestartet wird:
\begin{lstlisting}
if not "%1"=="am_admin" (powershell start -verb runas '%0' am_admin & exit)
\end{lstlisting}
Öffnen einer Konsole als Systemnutzer (muss als Administrator ausgeführt werden):
\begin{lstlisting}
PsExec.exe -i -s -d CMD
\end{lstlisting}
Erlaube Ausführung von Powershell-Skripten:
\begin{lstlisting}
C:\Windows\SysWOW64\WindowsPowerShell\v1.0\powershell.exe Set-ExecutionPolicy -Scope "LocalMachine" -ExecutionPolicy "Unrestricted"
\end{lstlisting}
Erlaube RDP-Verbindungen:
\begin{lstlisting}
REG.exe ADD "HKLM\SYSTEM\CurrentControlSet\Control\Terminal Server" /f /v fDenyTSConnections /t REG_DWORD /d 0
\end{lstlisting}
Schalte das Speichern von Thumbnails aus:
\begin{lstlisting}
Windows Registry Editor Version 5.00

[HKEY_CURRENT_USER\Software\Microsoft\Windows\CurrentVersion\Policies\Explorer]
"NoThumbnailCache"=dword:00000001
"DisableThumbnailCache"=dword:00000001

[HKEY_CURRENT_USER\Software\Policies\Microsoft\Windows\Explorer]
"DisableThumbsDBOnNetworkFolders"=dword:00000001

[HKEY_LOCAL_MACHINE\SOFTWARE\Microsoft\Windows\CurrentVersion\Policies\Explorer]
"NoThumbnailCache"=dword:00000001
"DisableThumbnailCache"=dword:00000001

[HKEY_CURRENT_USER\Software\Microsoft\Windows\CurrentVersion\Explorer\Advanced]
"DisableThumbnailCache"=dword:00000001
"NoThumbnailCache"=dword:00000001

[HKEY_LOCAL_MACHINE\SOFTWARE\Microsoft\Windows\CurrentVersion\Explorer\Advanced]
"DisableThumbnailCache"=dword:00000001
"NoThumbnailCache"=dword:00000001
\end{lstlisting}

\section{Fast Startup und Ruhezustand}
Datei: \path{hiberfil.sys}
\subsubsection{Zustände}
\settowidth{\MyLen}{\texttt{volume.id.space.l}}
\begin{tabular}{@{}p{\the\MyLen}%
		@{}p{\linewidth-\the\MyLen}@{}}
	\texttt{HIBR} & Im Ruhezustand\\
	\texttt{RSTR} & Wird fortgesetzt\\
	\texttt{WAKE} & Nach Fortsetzung\\
\end{tabular}

\subsubsection{Forensische Bewertung}
Änderung des Formats ab Win8
\begin{itemize}[leftmargin=*]
	\item Header bleibt auch nach Fortsetzen verfügbar
	\item Daten nur zwischen Versetzen in Ruhezustand bis zur Fortsetzung
	\item Vor Win8 zeitlich weit zurückreichende Daten
	\item Sichern der hiberfil.sys im laufenden Zustand keine forensisch relevanten Daten
	\item Größte Menge Daten \texttt{shutdown /h}
	\item HIBR2BIN ermöglicht dekomprimieren der Daten im neuen Format
	\item Fast Startup liefert keine interessanten Daten, da alle Applikationen beendet sind
\end{itemize}

\section{Edge Browser / ESE-DB}

\subsubsection{Anwendungspfad}
\path{C:\Windows\SystemApps\Microsoft.MicrosoftEdge_8wekyb3d8bbwe\MicrosoftEdge}

\subsection{ESE-Datenbank}
\subsubsection{Transaktionsflow}
\settowidth{\MyLen}{\texttt{transaction.im.ram.anwenden.lc.arr.pc.space}}
\begin{tabular}{@{}p{\the\MyLen}%
		@{}}
	\texttt{1. Transaction in RAM (Log Cache)}\\
	\texttt{2. Seiten aus DB in RAM (Page Cache)}\\
	\texttt{3. Transaktion im RAM anwenden (LC$\rightleftarrows$PC)}\\
	\texttt{4. Aktualiserte Daten in Logdatei (LC$\rightarrow$Datei)}\\
	\texttt{5. Datenbank aktualisieren}\\
\end{tabular}

\subsubsection{Dirty-DB}
Datenbank, die nicht vollständig aktualisiert wurde.
\settowidth{\MyLen}{\texttt{v01.chk.space.l}}
\begin{tabular}{@{}p{\the\MyLen}%
		@{}p{\linewidth-\the\MyLen}@{}}
	\texttt{V01.chk} & Zeitpunkt der Transaktion\\
	\texttt{*.log} & Transaktionsdaten, hexadezimale Dateinamen\\
\end{tabular}
\textbf{Wiederherstellung mit esentutl}\\
\settowidth{\MyLen}{\texttt{esentutl.mh.database.dat.space}}
\begin{tabular}{@{}p{\the\MyLen}%
		@{}p{\linewidth-\the\MyLen}@{}}
	\texttt{esentutl /mh database.dat} & Überprüfung der Datenbank (Feld State=Dirty)\\
	\texttt{esentutl /r database.dat} &  Reparatur der Datenbank (Feld State=Clean)\\
\end{tabular}

\subsection{WebCacheV01.dat}
\subsubsection{Pfade}
$\rightarrow$\path{C:\Users\<username>\AppData\Local\Packages\Microsoft.MicrosoftEdge_8wekyb3d8bbwe\AC\MicrosoftEdge\} (enthält v.a. Verweise und Speicherorte)
$\rightarrow$\path{C:\Users\<username>\AppData\Local\Packages\Microsoft.MicrosoftEdge_8wekyb3d8bbwe\AC\#!<number>\MicrosoftEdge\}

\subsubsection{Aufbau}
\settowidth{\MyLen}{\texttt{tabelle.container.n.spa}}
\begin{tabular}{@{}p{\the\MyLen}%
		@{}p{\linewidth-\the\MyLen}@{}}
	\textbf{Tabelle \texttt{Containers}} & \\
	\texttt{ContainerId} & Referenz auf Tabelle \texttt{Container\_n}\\
	\texttt{Directory} & Pfad zum Verzeichnis mit zwischengespeicherten Daten\\
	\texttt{SecureDirectories} & Zufällige Zeichenfolge, in 8er-Gruppen teilbar\\
	\texttt{Name} & Containertyp (Cookies|Content|History|...)\\
	\texttt{PartitionId} & Integritätslevel, (Protected= Internet=Low | lokal=medium)\\
	\textbf{Tabelle \texttt{Container\_n}} & \\
	\texttt{SecureDirectory} & Unterverzeichnis im Cachepfad \\
	\texttt{Type} & z.B. "In Private"-Modus (siehe Chivers)\\
	\texttt{AccessCount} & Anzahl wie oft URL referenziert wird\\
	<Timestamps> & \texttt{Sync, Creation, Expiry, Modified, Accessed Time}\\
	\texttt{URL} & Quelle der Informationen\\
	\texttt{Filename} & Name der Cachedatei\\
\end{tabular}

\subsubsection{Cache-Speicherort ermitteln}
\settowidth{\MyLen}{\texttt{securedirectories.spa}}
\begin{tabular}{@{}p{\the\MyLen}%
		@{}p{\linewidth-\the\MyLen}@{}}
	\texttt{SecureDirectories} & in 8er-Blöcke aufteilen\\
	\texttt{SecureDirectory} & zeigt auf x-ten Block (in Container\_n)\\
	\texttt{Directory} & Zeichenfolge anhängen\\
\end{tabular}

\subsubsection{Zeitstempel}
\settowidth{\MyLen}{\texttt{creationtime.spa}}
\begin{tabular}{@{}p{\the\MyLen}%
		@{}p{\linewidth-\the\MyLen}@{}}
	\texttt{CreationTime} & Erstellungzeit der Cachedatei/-objekt\\
	\texttt{ExpiryTime} & vom Webserver vorgegeben, Cache wird ungültig\\
	\texttt{ModifiedTime} & vom Webserver, Zeitpunkt der letzten Änderung der Ressource\\
	\texttt{AccessTime} & Letzter Zugriff des Nutzers auf Datei\\
\end{tabular}

\subsubsection{Werkzeuge}
Fazit: Tools gute Unterstützung, manuell bringt mehr
\settowidth{\MyLen}{\texttt{browserhistoryview.s}}
\begin{tabular}{@{}p{\the\MyLen}%
		@{}p{\linewidth-\the\MyLen}@{}}
	\texttt{IECacheView} & Zeigt Cachedateien von IE und Edge (Dateiname, -größe, -typ, URL, Zeitstempel, Cachedateipfad)\\
	\texttt{BrowsingHistoryView} & Zeigt Browserverlauf mehrerer Browser\\
\end{tabular}

\section{OneDrive}
\subsubsection{Anwendungspfad}
\path{C:\User\<username>\AppData\Local\Microsoft\OneDrive\}

\subsubsection{Registry}
\path{HKU\Software\Microsoft\OneDrive\}
\settowidth{\MyLen}{\texttt{....accounts.personal.spa}}
\begin{tabular}{@{}p{\the\MyLen}%
		@{}p{\linewidth-\the\MyLen}@{}}
	\path{.\} & \texttt{Version,UserFolder}\\
	\path{.\Accounts\Personal} & \texttt{ClientFirstSignInTimestamp, UserCID, UserFolder}\\
\end{tabular}

\subsubsection{Konfigurations- und Diagnostikdaten}
Ausgehend vom One-Drive-Verzeichnis:
\settowidth{\MyLen}{\texttt{..logs.personal.synclog.s}}
\begin{tabular}{@{}p{\the\MyLen}%
		@{}p{\linewidth-\the\MyLen}@{}}
	\path{.\logs\Personal\SyncDiagnostics.log} & Down-\textbackslash Uploadgeschwindigkeit, Ausstehende Down-\textbackslash Uploads, verfügbarer Speicherplatz lokal, UserCID (siehe REG), Anzahl Dateien und Verzeichnisse\\
	\path{.\settings\Personal\<usercid>.dat} & bisher kein Parser, mit Hexeditor Dateinamen einsehen\\
	\path{.\settings\Personal\<uploads|downloads>.txt} & Während Download temporär Daten wie Dateiname und User-CID\\
\end{tabular}

\subsubsection{Logdateien}
\path{.\logs\Personal\}\\
\path{*.aodl, *.odlsent, *.odl} enthalten Clientaktivitäten\\
Die Datei \path{ObfuscationStringMap.txt} enthält verschleierte Dateinamen, die in den Logs gefunden werden können. Mögliche Aktionen in den Logs:
\settowidth{\MyLen}{\texttt{....accounts.personal.spa}}
\begin{tabular}{@{}p{\the\MyLen}%
		@{}p{\linewidth-\the\MyLen}@{}}
	\texttt{FILE\_ACTION\_ADDED} & Datei lokal hinzugefügt\\
	\texttt{FILE\_ACTION\_REMOVED} & Datei lokal entfernt\\
	\texttt{FILE\_ACTION\_RENAMED} & Datei umbenannt\\
\end{tabular}

\subsubsection{Arbeitsspeicher}
Username und Passwort liegen im Klartext vor, nach Parameter \texttt{\&passwd=} und \texttt{\&loginmft=} suchen\\

\section{Benachrichtigungen und Kacheln}
\subsubsection{Datenbank}
\path{C:\Users\<username>\AppData\Local\Microsoft\Windows\Notifications}
\settowidth{\MyLen}{\texttt{....accounts.personal.spa}}
\begin{tabular}{@{}p{\the\MyLen}%
		@{}p{\linewidth-\the\MyLen}@{}}
	\texttt{wpndatabase.db} & Datenbank (Signatur \texttt{53 51 4C 69 74 65 20 66 6F 72 6D 61 74 20 33})\\
	\texttt{wpndatabase.db-wal} & Writhe Ahead Log (Signatur \texttt{37 7F 06 82} oder \texttt{37 7F 06 83})\\
	\texttt{wpndatabase.db-shm} & Shared Memory File, keine spezifische Signatur\\
\end{tabular}

SQLite-Datenbank mit WAL-Verfahren: Änderungen in Datei, bei Erreichen des Checkpoints (manuell oder automatisch) synchronisiert. WAL-Dateien bei der Untersuchung einbeziehen (\texttt{PRAGMA wal\_checkpoint}).

\subsubsection{Struktur und Inhalt}
Relevante Tabellen in \texttt{wpndatabase.db}
\settowidth{\MyLen}{\texttt{NotificationHandler.spa}}
\begin{tabular}{@{}p{\the\MyLen}%
		@{}p{\linewidth-\the\MyLen}@{}}
	\texttt{NotificationHandler} & Anwendungen, die zu Benachrichtigungen berechtigt sind (Zuordnung über PrimaryID$\rightarrow$ AppID,GUID)\\
	\texttt{Notification} & Benachrichtigunginhalt $\rightarrow$ Payload\\
\end{tabular}

\subsubsection{Kacheln}
Datenbank wie Benachrichtigungen,
Zeitstempel \texttt{ArrivalTime} und \texttt{ExpiryTime} Rückschlüsse auf Verwendung des Computers\\
Einige Anwendungen legen in dem DB-Verzeichnis Cacheordner an, die sehr lange zurückreichen

\section{Cortana}
\path{%localAppData%\Packages\Microsoft\Microsoft.Windows.Cortana_cw5n1h2txyewy}

\subsubsection{Artefakte}
\settowidth{\MyLen}{\texttt{....accounts.personal.spa}}
\begin{tabular}{@{}p{\the\MyLen}%
		@{}p{\linewidth-\the\MyLen}@{}}
	$\rightarrow$\path{.\AppData\Indexed DB\IndexedDB.edb} & 11 Tabellen, Tabelle \texttt{HeaderTable} enthält \texttt{createdTime, lastOpenTime}\\
	$\rightarrow$\path{.\LocalState\ESEDatabase_CortanaCoreInstance\CortanaCoreDb.dat} & [Veraltet] \texttt{Geofences} mit Standortdaten, \texttt{Reminders} benutzerspezifische Erinnerungen, \texttt{Triggers} LocationTriggers, TimeTriggers, ContactTriggers\\
	$\rightarrow$\path{.\LocalState\DeviceSearchCache\} & keine Dokumentation, Infos über Programmeinträgen, -aufrufen, Zeitstempel und JL-Einträge\\
	$\rightarrow$\path{.\AC\INetCache\<randomnumber>} & vollständige HTML-Seite von Suchen über Cortana\\
	$\rightarrow$\path{.\AC\AppCache\<randomnumber>} & HTML- und JavaScript Dateien für Cortana-Suche\\
	$\rightarrow$\path{.\LocalState\LocalRecorder\Speech} & Aufgezeichnete Sprachbefehle\\
	$\rightarrow$\path{.\LocalState\Cortana\Uploads\Contacts} & Falls Synchronisierung mit Android, Kontaktdaten und Mobilnummern\\
	$\rightarrow$\path{9d1f905ce5044aee.automaticDestinations-ms} & URLs die über Cortane-Suche ausgelöst wurden\\
	$\rightarrow$\path{WebCacheV01.dat} & URLs die über Cortana aufgerufen wurden\\
	$\rightarrow$\path{%SystemDrive%\Windows\Prefetch\SEARCHUI.EXE-14F7ADB7.pf} & Letzte Ausführungszeit(en)\\
	$\rightarrow$\path{%SystemDrive%\Windows\appcompat\Programs\Amcache.hve} & Erstellungs- und Änderungszeitstempel der Anwendung\\
\end{tabular}

\subsubsection{Deaktivieren von Cortana}
Parameter in \path{HKLM\Software\Policies\Microsoft\Windows\Windows Search}
\settowidth{\MyLen}{\texttt{....accounts.personal.spa}}
\begin{tabular}{@{}p{\the\MyLen}%
		@{}p{\linewidth-\the\MyLen}@{}}
	\texttt{AllowCortana} & \texttt{dword:00000000}\\
	\texttt{DisableWebSearch} & \texttt{dword:00000001}\\
	\texttt{AllowSearchToUseLocation} & \texttt{dword:00000000}\\
	\texttt{ConnectedSearchUseWeb} & \texttt{dword:00000000}\\
	\texttt{ConnectedSearchPrivacy} & \texttt{dword:00000003}\\
\end{tabular}

\end{multicols}
