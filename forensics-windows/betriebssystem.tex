\raggedright
\footnotesize
\begin{multicols}{3}	
	% multicol parameters
	% These lengths are set only within the two main columns
	%\setlength{\columnseprule}{0.25pt}
	\setlength{\premulticols}{1pt}
	\setlength{\postmulticols}{1pt}
	\setlength{\multicolsep}{1pt}
	\setlength{\columnsep}{2pt}

\begin{center}
     \Large{\underline{Betriebssystemforensik (allgemein)}} \\
     \small by Koll, Michael \\
     \url{https://github.com/michkoll/}
\end{center}

\section{Betriebssystem}
\subsection{Architektur}
\subsubsection{Monolithisch (S.22)}
\settowidth{\MyLen}{\texttt{speichereffizienz.spa}}
\begin{tabular}{@{}p{\the\MyLen}%
		@{}p{\linewidth-\the\MyLen}@{}}
	\texttt{Geschwindigkeit} & schnell, minimaler Overhead; Funktionen optim. abgestimmt\\
	\texttt{Sicherheit} & Risiko: ganzes BS im priv. Modus; Probleme einzerln Komp. Auswirkung auf ganzes BS\\
	\texttt{Speichereffizienz} & Schlecht, ganzes BS im Speicher gehalten\\
	\texttt{Wartbarkeit, Erweiterbarkeit} & Schlecht, da bei Änderungen viele Komponenten\\
\end{tabular}

\subsubsection{Geschichtet (S.23)}
\settowidth{\MyLen}{\texttt{speichereffizienz.spa}}
\begin{tabular}{@{}p{\the\MyLen}%
		@{}p{\linewidth-\the\MyLen}@{}}
	\texttt{Geschwindigkeit} & Langsamer, da Funktionen Overhead, häufiger Kontextwechsel\\
	\texttt{Sicherheit} & Teile des BS im User Mode, z.B. Treiber; Probleme Komponenten $\nrightarrow$ BS\\
	\texttt{Speichereffizienz} & Gut, einzelne Module dynamisch nachgeladen und entladen\\
	\texttt{Wartbarkeit, Erweiterbarkeit} & Besser, da Änderungen meist nur bei einzelnen Komponenten\\
\end{tabular}

\subsubsection{Mikrokernel (S.24)}
\settowidth{\MyLen}{\texttt{speichereffizienz.spa}}
\begin{tabular}{@{}p{\the\MyLen}%
		@{}p{\linewidth-\the\MyLen}@{}}
	\texttt{Geschwindigkeit} & schlechte Performance, häufige Prozesswechsel und Interprozesskommunikation\\
	\texttt{Sicherheit} & sicherheitskritischer Teil relativ klein; Dienste außerhalb Kern können Sicherheit und Stabilität nicht beeinflussen\\
	\texttt{Speichereffizienz} & Gut, einzelne Module dynamisch nachgeladen und entladen\\
	\texttt{Wartbarkeit, Erweiterbarkeit} & Sehr gut, einzelne Module können ausgetauscht werden (z.T. während Betrieb)\\
\end{tabular}

\subsection{Ziele}
%\settowidth{\MyLen}{\texttt{speichereffizienz.spa}}
\begin{tabular}{@{}p{\the\MyLen}@{}}
	\texttt{}
\end{tabular}



\section{Windows}

\subsection{Allgemein}
\subsubsection{Windows Stations, Desktops und Session (S.34)}
Authentifizierung Session-orientiert, \texttt{Session} beinhaltet mehrere \texttt{Stations}, \texttt{Stations} beinhalten Desktops mit Fenstern und GDI-Objekten. Sicherheitsbeschreiber eines Objekts ist mit \texttt{Station} verbunden, darüber Kontrolle von Benutzer zum Desktop

\subsection{Prozesse und Dienste}
\subsubsection{svchost.exe (Dienste) (S.138)}
\begin{itemize}[leftmargin=*]
	\item mit \texttt{tlist} laufende Prozesse mit Diensten auflisten (\texttt{tlist -m svchost.exe -s})
	\item mit \texttt{Process-Explorer} farblich gekennzeichnete Dienste $\rightarrow$ Properties $\rightarrow$ Services
	\item spezielle Programme wie z.B. \texttt{svchost-Analyzer}
\end{itemize}

\subsubsection{Gestartete Dienste in Registry}
\path{HKLM\System\CurrentControlSet\Services} als Unterschlüssel

\subsubsection{Mandatorische Zugriffsregeln (S.153)}
\settowidth{\MyLen}{\texttt{breite.erste.spalte}}
\begin{tabular}{@{}p{\the\MyLen}%
		@{}p{\linewidth-\the\MyLen}@{}}
	\texttt{No-<Write|Read>-Up} & Kein schreibender/lesender Zugriff von Prozessen mit niedrigem Level auf Objekte mit höherem Level (gleiches Level zugelassen) \\
	\texttt{No-<Write|Read>-Down} & Kein schreibender/lesender Zugriff von Prozessen mit höherem Level auf Objekte mit niedrigerem Level (gleiches Level zugelassen) \\
\end{tabular}
\textbf{Default:} No-Write-Up (für alle Objekte), No-Read-Up (für Prozesse und Threads)

\subsection{DACL (S.156)}
Sicherheitsdeskriptor besteht aus \texttt{Header, SID Besitzer, SID Gruppe, DACL, SACL}\\
\texttt{DACL} besteht aus ACEs mit \texttt{<Allow|Deny>, SID User, ACE-Bitmapp}\\
\textbf{Regeln DACL}: Erst Einzel-ACE, dann Gruppe; Erst Verbote, dann Erlaubnisse; Reihenfolge von oben nach unten\\
\texttt{Hinweis:} Beim Ändern bzw. lesen aufpassen auf Gruppenzugehörigkeit (Jeder)





\subsection{Registry}






% You can even have references
\rule{0.3\linewidth}{0.25pt}
\scriptsize
\bibliographystyle{abstract}
\bibliography{refFile}
\end{multicols}