\raggedright
\footnotesize
\begin{multicols}{3}	
	% multicol parameters
	% These lengths are set only within the two main columns
	%\setlength{\columnseprule}{0.25pt}
	\setlength{\premulticols}{1pt}
	\setlength{\postmulticols}{1pt}
	\setlength{\multicolsep}{1pt}
	\setlength{\columnsep}{2pt}

\begin{center}
     \Large{\underline{Betriebssystemforensik (allgemein)}} \\
\end{center}

\section{Betriebssystem}
\subsection{Architektur}
\subsubsection{Monolithisch (S.22)}
\settowidth{\MyLen}{\texttt{speichereffizienz.spa}}
\begin{tabular}{@{}p{\the\MyLen}%
		@{}p{\linewidth-\the\MyLen}@{}}
	\texttt{Geschwindigkeit} & schnell, minimaler Overhead; Funktionen optim. abgestimmt\\
	\texttt{Sicherheit} & Risiko: ganzes BS im priv. Modus; Probleme einzerln Komp. Auswirkung auf ganzes BS\\
	\texttt{Speichereffizienz} & Schlecht, ganzes BS im Speicher gehalten\\
	\texttt{Wartbarkeit, Erweiterbarkeit} & Schlecht, da bei Änderungen viele Komponenten\\
\end{tabular}

\subsubsection{Geschichtet (S.23)}
\settowidth{\MyLen}{\texttt{speichereffizienz.spa}}
\begin{tabular}{@{}p{\the\MyLen}%
		@{}p{\linewidth-\the\MyLen}@{}}
	\texttt{Geschwindigkeit} & Langsamer, da Funktionen Overhead, häufiger Kontextwechsel\\
	\texttt{Sicherheit} & Teile des BS im User Mode, z.B. Treiber; Probleme Komponenten $\nrightarrow$ BS\\
	\texttt{Speichereffizienz} & Gut, einzelne Module dynamisch nachgeladen und entladen\\
	\texttt{Wartbarkeit, Erweiterbarkeit} & Besser, da Änderungen meist nur bei einzelnen Komponenten\\
\end{tabular}

\subsubsection{Mikrokernel (S.24)}
\settowidth{\MyLen}{\texttt{speichereffizienz.spa}}
\begin{tabular}{@{}p{\the\MyLen}%
		@{}p{\linewidth-\the\MyLen}@{}}
	\texttt{Geschwindigkeit} & schlechte Performance, häufige Prozesswechsel und Interprozesskommunikation\\
	\texttt{Sicherheit} & sicherheitskritischer Teil relativ klein; Dienste außerhalb Kern können Sicherheit und Stabilität nicht beeinflussen\\
	\texttt{Speichereffizienz} & Gut, einzelne Module dynamisch nachgeladen und entladen\\
	\texttt{Wartbarkeit, Erweiterbarkeit} & Sehr gut, einzelne Module können ausgetauscht werden (z.T. während Betrieb)\\
\end{tabular}

\subsubsection{Vorteile virtuelles BS}
Sandbox verbesserte Sicherheit durch Abschottung; bessere Ausnutzung des Systems durch mehrere VMs; herstellen kompatibler Laufzeitumgebungen\\

\subsection{Ziele (S.12)}
\settowidth{\MyLen}{\texttt{speichereffizienz.spa}}
\begin{tabular}{@{}p{\the\MyLen}%
		@{}p{\linewidth-\the\MyLen}@{}}
	\texttt{Unterstützung des Anwenders} & \texttt{Abstraktion der Hardware} (Nummerierte Datenblöcke der HDD werden durch Reihenfolge, Verkettung und Verknüpfung zu Datei), \texttt{Bereitstellen von Dienstfunktionen} (Dateien öffnen, lesen, schreiben, schließen), \texttt{Verbergen irrelevanter Details} (Nummerierung Datenblöcke für Anwender nicht sichtbar)\\
	\texttt{Optimierung der Rechnerauslastung} & Parallele Nutzung Rechnerkomponenten, mehrere Aufgaben quasiparallel\\
	\texttt{Zuverlässigkeit} & Schutzmechanismus gegenseitig störender Prozesse, Abfangen von Ausnahmesituationen, Verhindern von blockierenden Prozessen\\
	\texttt{Portabilität} & Programme auf verschiedenen Plattformen lauffähig\\
\end{tabular}
\textbf{Nicht erfüllte Zuverlässigkeit}\\
Prozess belegt zu viel Speicher, so dass andere Prozesse nicht ausgeführt werden können\\
Abbruch mit Ctrl+C funktioniert nicht, da Signal auf Ignorieren steht\\
Prozess zieht alle Prozessorleistung, so dass andere Prozesse blockiert sind (unfaires Scheduling)

\subsection{Aufgaben (S.14)}
\settowidth{\MyLen}{\texttt{Anwenderschnittstelle.spa}}
\begin{tabular}{@{}p{\the\MyLen}%
		@{}p{\linewidth-\the\MyLen}@{}}
	\texttt{Programm- und Prozessverwaltung} & Steuern, Erzeugen, Starten, Entfernen von Prozessen; Laden von Programmen von HDD in RAM; Leerlaufprozess; Kommunikation und Synchronisation von Prozessen\\
	\texttt{Anwenderschnittstelle} & Kommandoebene, graphische Bedienoberfläche, Systemaufrufe zwischen BS und Programmen\\
	\texttt{Verwalten von Betriebsmitteln} & Aufteilen der Betriebsmittel, Trennung Benutzerbereiche, Schutz, Prüfung Zugang\\
	\texttt{Verbindungen mit anderen Rechnern} & \\
\end{tabular}

\subsection{Begriffe}
\settowidth{\MyLen}{\texttt{quasiparallel.spa}}
\begin{tabular}{@{}p{\the\MyLen}%
		@{}p{\linewidth-\the\MyLen}@{}}
	\texttt{Parallel} & Gleichzeitige Abarbeitung von Prozessen, jeder Prozess läuft auf eigener CPU\\
	\texttt{Quasiparallel} & Abwechselnde Abarbeitung, alle Prozesse laufen auf gleicher CPU\\
	\texttt{Programm} & besteht aus Vorschriften/Anweisungen in formaler Sprache; Ausführen zur Bewältigung bestimmter Aufgaben\\
	\texttt{Prozess} & ablaufendes Programm mit konkreten Daten, besitzt \texttt{Rechte, Registerinhalte und Speicher}; Zustände \texttt{running, ready oder waiting}\\
	\texttt{Threads} & Untereinheit von Prozessen, teilen sich denselben virtuellen Adressraum, Prozesswechsel schneller\\
	\texttt{Leerlaufprozess} & Prozessor führt ständig Befehlszyklen aus, Leerlaufprozess verbraucht diese mit NOP-Anweisungen\\
\end{tabular}

\section{Dateisystem}
\subsubsection{Zusammenhängende Belegung (S.104)}
\settowidth{\MyLen}{\texttt{Belegungstabelle.spa}}
\begin{tabular}{@{}p{\the\MyLen}@{}%
		@{}p{\linewidth-\the\MyLen}@{}}
\texttt{Belegungstabelle} & Datei, Start, Länge\\
\end{tabular}

\subsubsection{Verteilte Belegung verkettete Listen (FAT) (S.105)}
\settowidth{\MyLen}{\texttt{Belegungstabelle.spa}}
\begin{tabular}{@{}p{\the\MyLen}@{}%
		@{}p{\linewidth-\the\MyLen}@{}}
	\texttt{Belegungstabelle} & Datei, Start\\
	\texttt{Hilfstabelle (FAT)} & Verweis auf nächste Adresse, Dateiende mit EOF\\
\end{tabular}

\subsubsection{Verteilte Belegung mittels Index-Liste (S.106)}
\settowidth{\MyLen}{\texttt{Belegungstabelle.spa}}
\begin{tabular}{@{}p{\the\MyLen}@{}%
		@{}p{\linewidth-\the\MyLen}@{}}
	\texttt{Belegungstabelle} & Datei, Index-DU\\
	\texttt{Index-DU} & Verweise auf DUs (falls zu lang Verweis auf weitere Index-DU)\\
\end{tabular}


\section{Windows}

\subsection{Allgemein}
\subsubsection{Windows Stations, Desktops und Session (S.34)}
Authentifizierung Session-orientiert, \texttt{Session} beinhaltet mehrere \texttt{Stations}, \texttt{Stations} beinhalten Desktops mit Fenstern und GDI-Objekten. Sicherheitsbeschreiber eines Objekts ist mit \texttt{Station} verbunden, darüber Kontrolle von Benutzer zum Desktop

\subsection{Prozesse und Dienste}
\subsubsection{svchost.exe (Dienste) (S.138)}
\begin{itemize}[leftmargin=*]
	\item mit \texttt{tlist} laufende Prozesse mit Diensten auflisten (\texttt{tlist -m svchost.exe -s})
	\item mit \texttt{Process-Explorer} farblich gekennzeichnete Dienste $\rightarrow$ Properties $\rightarrow$ Services
	\item spezielle Programme wie z.B. \texttt{svchost-Analyzer}
\end{itemize}

\subsubsection{Gestartete Dienste in Registry}
\path{HKLM\System\CurrentControlSet\Services} als Unterschlüssel

\subsubsection{laufende Prozesse PIDs und TIDs}
mit Process Explorer; \texttt{PID} in Liste laufende Prozesse; \texttt{TID} Prozesseigenschaften $\rightarrow$ Threads\\

\subsubsection{Registryzugriffe von Prozessen}
Mit Process Explorer und Process Hacker; Möglichkeit über Process Monitor Registryzugriffe zu protokollieren (Software installieren $\rightarrow$ mit Process Monitor analysieren)\\

\subsubsection{Ausgeführte Dienste}
z.B. über \texttt{msc} (services) oder Registry (siehe oben)

\subsubsection{Mandatorische Zugriffsregeln (S.153)}
\settowidth{\MyLen}{\texttt{No-<Write|Read>-Down.spa}}
\begin{tabular}{@{}p{\the\MyLen}%
		@{}p{\linewidth-\the\MyLen}@{}}
	\texttt{No-<Write|Read>-Up} & Kein schreibender/lesender Zugriff von Prozessen mit niedrigem Level auf Objekte mit höherem Level (gleiches Level zugelassen) \\
	\texttt{No-<Write|Read>-Down} & Kein schreibender/lesender Zugriff von Prozessen mit höherem Level auf Objekte mit niedrigerem Level (gleiches Level zugelassen) \\
\end{tabular}
\textbf{Default:} No-Write-Up (für alle Objekte), No-Read-Up (für Prozesse und Threads)

\subsection{DACL (S.156)}
Sicherheitsdeskriptor besteht aus \texttt{Header, SID Besitzer, SID Gruppe, DACL, SACL}\\
\texttt{DACL} besteht aus ACEs mit \texttt{<Allow|Deny>, SID User, ACE-Bitmapp}\\
\textbf{Regeln DACL}: Erst Einzel-ACE, dann Gruppe; Erst Verbote, dann Erlaubnisse; Reihenfolge von oben nach unten\\
\texttt{Hinweis:} Beim Ändern bzw. lesen aufpassen auf Gruppenzugehörigkeit (Jeder)

\subsection{Festplatten und Drucker}
\settowidth{\MyLen}{\texttt{option.2.spa}}
\begin{tabular}{@{}p{\the\MyLen}%
		@{}p{\linewidth-\the\MyLen}@{}}
	\texttt{Option 1} & In regedit \path{HKEY_LOCAL_MACHINE\SYSTEM} exportieren, in RegRipper Report erstellen\\
	\texttt{Option 2} & Systemwerkzeuge wie \texttt{msinfo} \\
\end{tabular}

\subsection{Forensische Anwendungsfälle}
\subsubsection{Suchen mit X-Ways}
\settowidth{\MyLen}{\texttt{Nach.Unicode.String.spa}}
\begin{tabular}{@{}p{\the\MyLen}%
		@{}p{\linewidth-\the\MyLen}@{}}
	\texttt{Nach Hexwert in Bild} & Image einbinden, Datei nach hex-Wert durchsuchen\\
	\texttt{Nach ASCII-String in Dokument} & Image einbinden, nach Text-Wert suchen mit ASCII-Codepage \\
	\texttt{Nach Unicode-String in Dokument} & Image einbinden, nach Text-Wert suchen mit Unicode-Codepage\\
	\texttt{in docx-Datei} & Image einbinden, Indexieren, Index nach Text-Wert durchsuchen mit ASCII- oder Unicode-Codepage
\end{tabular}

\subsubsection{Carving}
Carving-Programm durchsucht Dokument von Anfang nach Anfangssignatur, Markierung, Suchen Richtung Ende nach Endesignatur; Bereich dazwischen in Datei kopieren

\subsubsection{Schattenkopie}
\texttt{Volume-Shadow-Copy-Service (VSS)} hält Dateien in mehreren Versionen, Versionen können über Eigenschaften $\rightarrow$ Versionen eingesehen werden. Zur Analyse Schattenkopie mounten

\subsubsection{Thumbs.db}
Inhalte können mit \texttt{Thumb.db-Viewer} sichtbar gemacht werden (bildlich oder als Liste); Ungefähres Erscheinungsbild, Speicherort des Originals und Veränderungsdatum kann eingesehen werden

\subsubsection{Überwachter Ordnerzugriff}
(Details auf eigenem CheatSheet)\\
Angriffsmöglichkeiten prüfen, dazu:
\settowidth{\MyLen}{\texttt{Standardverzeichnisse.spa}}
\begin{tabular}{@{}p{\the\MyLen}%
		@{}p{\linewidth-\the\MyLen}@{}}
	\texttt{Ist überwachter Ordnerzugriff aktiviert?} &  Windows Defender, Registry oder Gruppenrichtlinien\\
	\texttt{Standardverzeichnisse} & Falls aktiviert, sind diese geschützt\\
	\texttt{Zusätzliche Verzeichnisse} & Schauen ob Verzeichnis hinzugefügt (in Registry oder Windows Defender)\\
	\texttt{Erlaubte Anwendungen} & Schauen ob Anwendungen erlaubt sind (in Registry)\\
\end{tabular}

\subsubsection{Nutzung OneDrive}
Anhaltspunkte zur Nutzung\\
\settowidth{\MyLen}{\texttt{ClientFirstSignInTimestamp.spa}}
\begin{tabular}{@{}p{\the\MyLen}%
		@{}p{\linewidth-\the\MyLen}@{}}
	\texttt{UserFolder} & Schauen ob vorhanden\\
	\texttt{ClientFirstSignInTimestamp} & Erster Login des Nutzers\\
	\texttt{UserCID} & Falls vorhanden muss genutzt worden sein\\
	\texttt{Logdateien} & Infos zu Anzahl Dateien, Up-/Downloadgeschwindigkeit, UserCID\\
\end{tabular}

\section{UNIX}
\subsection{Systemzustand}
Werkzeuge verwenden Informationen aus \path{/proc}-Verzeichnis\\
\settowidth{\MyLen}{\texttt{Speicherauslastung.spa}}
\begin{tabular}{@{}p{\the\MyLen}%
		@{}p{\linewidth-\the\MyLen}@{}}
	\texttt{Uptime} & \path{/proc/cpuinfo}\\
	\texttt{Systemauslastung} & \path{/proc/stat}\\
	\texttt{Speicherauslastung} & \path{/proc/meminfo}\\
	\texttt{Version BS} & \path{/proc/version} \\
	\texttt{Dateisysteme} & \path{/proc/filesystem} \\
\end{tabular}

\end{multicols}