\raggedright
\footnotesize
\begin{multicols}{3}	
	\setlength{\premulticols}{1pt}
	\setlength{\postmulticols}{1pt}
	\setlength{\multicolsep}{1pt}
	\setlength{\columnsep}{2pt}

\begin{center}
     \Large{\underline{Kryptographie}} \\
\end{center}

\section{Arten von Kryptographie}
\subsection{Symmetrische Kryptographie}
Bei der symmetrischen Kryptographie wird beim Ver- und Entschlüsseln der gleiche (geheime) Schlüssel verwendet. Wird eine symmetrische Verschlüsselung zur Kommunikation verwendet, müssen daher alle Teilnehmer den Schlüssel kennen. Es muss darauf vertraut werden, dass die anderen Teilnehmer den Schlüssel (weder gewollt noch ungewollt) weitergeben.
Beispiele für symmetrische Verschlüsselungsalgorithmen sind AES, DES, Blowfish und Serpent.
\subsection{Asymmetrische Kryptographie}
Bei asymmetrischen Verschlüsselungsverfahren gibt es einen privaten und einen öffentlichen Schlüssel. Mit beiden Schlüsseln kann man eine Nachricht verschlüsseln und anschließend mit dem jeweils anderen Schlüssel entschlüsseln. Der private Schlüssel darf bei Kommunikation mit asymmetrischen Verschlüsselungsverfahren niemals an irgendwen weiter gegeben werden, während der öffentliche Schlüssel an weitere Personen weiter gegeben werden kann/darf/muss, da dies technisch erforderlich ist.
\section{Benutzung von asymmetrischer Kryptographie}
%TODO erklärung einfügen, mit welchen commandline-aufrufen man keypairs generieren kann
\subsection{Verschlüsselte Kommunikation}
Bei verschlüsselter Kommunikation mittels eines asymmetrischen Kryptographie-Systems wird der öffentliche Schlüssel zur Verschlüsselung und der private Schlüssel zur Entschlüsselung einer Nachricht benutzt.\\
Beispiel:\\
A möchte eine Nachricht an B senden. B hat sich dafür ein Schlüsselpaar (bestehend auf öffentlichem und privatem Schlüssel) erzeugt. A verschlüsselt die Nachricht mit dem öffentlichen Schlüssel von B.
\subsection{Signatur}

\end{multicols}