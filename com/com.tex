\raggedright
\footnotesize
\begin{multicols}{3}	
	\setlength{\premulticols}{1pt}
	\setlength{\postmulticols}{1pt}
	\setlength{\multicolsep}{1pt}
	\setlength{\columnsep}{2pt}

\begin{center}
     \Large{\underline{COM}} \\
\end{center}
\section{COM}

\subsection{Was ist das Common Object Model (COM)?}
\lipsum[1]

\subsubsection{Wie funktioniret das COM?}
\lipsum[1]

\paragraph{Mehrfachvererbung}
\lipsum[1]

\subsubsection{Zyklische Referenzen}
\lipsum[1]

\subsection{COM Threading Model}
\lipsum[1]

\subsection{Datentypen}
\lipsum[1]

\subsubsection{Reguläre Datentypen}
\lipsum[1]

\subsubsection{Semireguläre Datentypen}
\lipsum[1]

\subsection{COM Interop}
\lipsum[1]

\subsection{OLE}
\lipsum[1]

\subsection{COM+}
\lipsum[1]

\subsection{DCOM}
\lipsum[1]

\subsection{COMDAT}
\lipsum[1]
\subsection{COFF}
\lipsum[1]
\subsection{OMF}
\lipsum[1]
\section{Diverses}

\subsection{Was ist das Common Object File Format?}
\lipsum[1]

\subsection{Thread-Apartments}

\subsubsection{STA-Thread}
\lipsum[1]

\subsubsection{MTA-Thread}
\lipsum[1]

\subsection{STL-Container}
\lipsum[1]

\subsection{Marshaling}
\lipsum[1]

\subsection{Binding}
\lipsum[1]

\subsection{Linking}
\lipsum[1]

\subsection{PE-Format}
\lipsum[1]

\subsection{POSIX}
\lipsum[1]

\subsection{thunk}
\lipsum[1]
\subsection{CRT}
\lipsum[1]

\subsection{Cast}

\subsubsection{static_cast}
\lipsum[1]
\subsubsection{dynamic_cast}
\lipsum[1]
\subsubsection{const_cast}
\lipsum[1]
\subsubsection{reinterpret_cast}
\lipsum[1]
\subsection{Zeiger}
\subsubsection{Smart Pointer}
\subsection{POD-Typen}

\subsection{ELF-Format}
\subsection{CLR}


\end{multicols}