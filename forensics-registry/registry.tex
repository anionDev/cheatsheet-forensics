%Preamble for every 
\raggedright
\footnotesize
\begin{multicols}{3}	
	% multicol parameters
	% These lengths are set only within the two main columns
	%\setlength{\columnseprule}{0.25pt}
	\setlength{\premulticols}{1pt}
	\setlength{\postmulticols}{1pt}
	\setlength{\multicolsep}{1pt}
	\setlength{\columnsep}{2pt}

\begin{center}
     \Large{\underline{Registryforensik}} \\
\end{center}

\section{Relative Pfade}
\settowidth{\MyLen}{\texttt{letterpaper}/\texttt{a4paper} \ }
\begin{tabular}{@{}p{\the\MyLen}%
				@{}p{\linewidth-\the\MyLen}@{}}
\texttt{\%UserProfile\%} & Pfad zum derzeitigen Benutzerprofil \\
\texttt{\%SystemDrive\%} & Laufwerksbuchstabe, auf dem Windows 
							installiert ist, i.d.R \texttt{C:} \\
\texttt{\%SystemRoot\%} & Pfad zum Windows Ordner, i.d.R. \texttt{C:\textbackslash Windows}
\end{tabular}	

\section{Schlüssel \& Werte}
Ein Schlüssel enthält einen oder mehrere Werte sowie einen Zeitstempel des letzten Zugriffs\\
Jeder Wert hat 3 Felder:
\settowidth{\MyLen}{value.fields.}
\begin{tabular}{@{}p{\the\MyLen}%
		@{}p{\linewidth-\the\MyLen}@{}}
	\texttt{Name} & Eindeutig innerhalb eines Schlüssels\\
	\texttt{Typ} & Datentyp des Wertes (s.u.)\\
	\texttt{Daten} & kann leer oder null sein, Maximum 32767 Bytes, häufig in hexadezimaler Notation\\
\end{tabular}
Die wichtigsten Datentypen sind
\settowidth{\MyLen}{Reg.Expand.leng}
\begin{tabular}{@{}p{\the\MyLen}%
		@{}p{\linewidth-\the\MyLen}@{}}
	\texttt{REG\_NONE} & kein definierter Typ\\
	\texttt{REG\_SZ} & Fixe Länge und NULL-Char am Ende\\
	\texttt{REG\_EXPAND\_SZ} & Variable Länge und NULL-Char am Ende\\
	\texttt{REG\_BINARY} & Binärdaten\\
	\texttt{REG\_DWORD} & Double-Word-Werte, häufig boolesche Werte\\
	\texttt{REG\_LINK} & Link\\
	\texttt{REG\_MULTI\_SZ} & Liste von Strings\\
\end{tabular}


\section{Struktur}
\subsection{Wurzelschlüssel}
\settowidth{\MyLen}{\texttt{HKCR.sp.}}
\settowidth{\MyLenTwo}{\texttt{HKEY\_CURRENT\_USER.space.enough.}}
\begin{tabular}{@{}p{\MyLen}%
				@{}p{\MyLenTwo}%
				@{}p{\linewidth-\the\MyLen-\the\MyLenTwo}@{}}
\texttt{HKLM} & HKEY\_LOCAL\_MACHINE & Hauptschlüssel \\
\texttt{HKU} & HKEY\_HKU & Hauptschlüssel \\
\texttt{HKCR} & HKEY\_CLASSES\_ROOT & Verweis\\
\texttt{HKCU} & HKEY\_CURRENT\_USER & Verweis\\
\texttt{HKCC} & HKEY\_CURRENT\_CONFIG & Verweis \\
\end{tabular}

\subsection{Verweise}
\settowidth{\MyLen}{HKCR.sp }
\begin{tabular}{@{}p{\the\MyLen}%
		@{}p{\linewidth-\the\MyLen}@{}}
\texttt{HKCC} & \texttt{HKLM\textbackslash SYSTEM\textbackslash CurrentControlSet\textbackslash Hardware Profiles\textbackslash Current} \\
\texttt{HKCU} & \texttt{HKU\textbackslash S-1-5-21-xxx} (SID)\\
\texttt{HKCR} & \texttt{HKLM\textbackslash SOFTWARE\textbackslash Classes}
\end{tabular}

\subsection{HKU}
Nutzerspezifische Einstellungen und Informationen für jeden aktiv geladenen Benutzer (Standardprofile und angemeldete Profile, keine abgemeldeten Nutzer)
\settowidth{\MyLen}{SID.classes.length.long.}
\begin{tabular}{@{}p{\the\MyLen}%
		@{}p{\linewidth-\the\MyLen}@{}}
	\texttt{.DEFAULT} & Einstellungen, die Windows nutzt, bevor ein Nutzer sich eingeloggt hat \\
	\texttt{S-1-5-18} & well-known SID für LocalSystem-Benutzer\\
	\texttt{S-1-5-19} & well-known SID für LocalService-Benutzer, lokale Dienste, die den LocalSystem-User nicht benötigen\\
	\texttt{S-1-5-20} & well-known SID für NetworkService-Benutzer, Netzwerkdienste, die den LocalService-Benutzer nicht benötigen\\
	\texttt{S-1-5-21-[...]} & SID des derzeit angemeldeten Benutzers (Link von \texttt{HKCU})\\
	\texttt{S-1-5-21-[...]\_Classes} & Nutzerspezifische Dateiverknüpfungen\\
\end{tabular}

\subsection{HKCU}
Link auf \texttt{HKU\textbackslash [SID]}\\
Spezifische Einstellungen und Informationen zum angemeldeten Benutzer (Umgebungsvariablen, Desktopeinstellungen, Netzwerkverbindungen, Drucker und Präferenzen)
\settowidth{\MyLen}{SID.classes.length.long.}
\begin{tabular}{@{}p{\the\MyLen}%
		@{}p{\linewidth-\the\MyLen}@{}}
	\texttt{AppEvents} & Verknüpft Audiodateien mit Aktionen (z.B. Ton beim Öffnen eines Menüs)\\
	\texttt{Console} & Daten zum Console-Subsystem (z.B. zum MS-DOS-Command-Prompt)\\
	\texttt{Control-Panel} & Einstellungen der Systemsteuerung, u.a. regionale Einstellungen und Erscheinungsbild\\
	\texttt{Environment} & Umgebungsvariablen, die Benutzer gesetzt haben\\
	\texttt{Keyboard-Layout} & Installierte Tastaturlayouts\\
	\texttt{Network} & Jeder Unterschlüssel ein Netzlaufwerk, Name des Schlüssels ist Laufwerksbuchstabe, enthält Konfigurationsdaten zum Verbinden\\
	\texttt{Printers} & Präferenzen des Benutzers zum Drucken\\
	\texttt{Software} & Nutzerspezifische Einstellungen zu installierten Programmen, je nach Programm Informationen zu Programmanbieter, Programm, Version, Installationsdatum und zulegt zugegriffene Dateien. Ablage nach HKCU\textbackslash Software\textbackslash Programmanbieter\textbackslash -Programm\textbackslash Version\\
	\texttt{Volatile Environment} & Umgebungsvariablen, die beim Login definiert wurden\\
\end{tabular}

\subsection{HKLM}
Spezifische Einstellugen des lokalen Rechners, die für alle Benutzer geladen werden.
\settowidth{\MyLen}{HLKM.folder.}
\begin{tabular}{@{}p{\the\MyLen}%
		@{}p{\linewidth-\the\MyLen}@{}}
	\texttt{HARDWARE} & Speichert HW-Daten beim Systemstart, wird bei jedem Start erstellt und mit Informationen über Geräte, Treiber und Ressourcen gefüllt\\
	\texttt{SAM} & Lokale Windows-Sicherheitsdatenbank über Benutzer- und Gruppeninformationen (Link zu \texttt{HKLM\textbackslash SECURITY\textbackslash SAM})\\
	\texttt{SECURITY} & Lokale Windows-Sicherheitsdatenbank (inklusive SAM)\\
	\texttt{SOFTWARE} & Einstellungen zu Applikationen des Rechners (und Microsoft-Applikationen)\\
	\texttt{SYSTEM} & Informationen zur Systemkonfiguration (z.B. Gerätetreiber und Dienste). Derzeitiges Hardwareprofil ist Link von \texttt{HKCC}. Mehrere Sätze mit Schema \texttt{ControlSetxxx}. \texttt{HKLM\textbackslash SYSTEM\textbackslash Select} zeigt aktuelle verwendetes Profil in \texttt{CurrentControlSet}.
\end{tabular}

\subsection{HKCR}
Link auf \texttt{HKLM\textbackslash Software\textbackslash Classes \& HKU\textbackslash [SID]\_Classes}
\begin{itemize}[leftmargin=*]
	\item Zuweisungen für Dateierweiterungen
	\item OLE-Datenbank
	\item Einstellungen für registrierte Anwendungen für COM-Objekte
	\item Nutzer- und systembasierte Informationen
\end{itemize}
Setzt sich aus \texttt{HKLM\textbackslash SOFTWARE\textbackslash Classes} und \texttt{HKU\textbackslash [SID]\_Classes} zusammen. Falls identischer Wert, hat \texttt{HKCU} Priorität.\\
Beispiel: Was soll passieren, wenn eine .pptx-Datei geöffnet wird. \texttt{HKCR} macht einen erheblichen Teil der Registry und des Systemverhaltens aus

\subsection{HKCC}
Link auf \texttt{HKLM\textbackslash System\textbackslash CurrenControlSet\textbackslash Hardware Profiles\textbackslash Current}\\
Link zu den Konfigurationsdaten des derzeitigen Hardwareprofils. Informationen werden bei jedem Booten neu erzeugt und daher nicht physisch in der Registry-Datei gespeichert.
\settowidth{\MyLen}{HLKM.folder.}
\begin{tabular}{@{}p{\the\MyLen}%
		@{}p{\linewidth-\the\MyLen}@{}}
	\texttt{System} & \\
	\texttt{Software} & \\
\end{tabular}

\section{Hives}
User-Profile-Hives in \texttt{\%UserProfile\%\textbackslash NTUSER.DAT}\\
Alle anderen Hives und Dateien in \%SystemRoot\%\textbackslash System32\textbackslash config
\settowidth{\MyLen}{\texttt{HKLM.SECURITY.len}}
\begin{tabular}{@{}p{\the\MyLen}%
		@{}p{\linewidth-\the\MyLen}@{}}
	\texttt{HKU\textbackslash .DEFAULT} & DEFAULT\\
	\texttt{HKLM\textbackslash SAM} & SAM\\
	\texttt{HLKM\textbackslash SECURITY} & SECURITY\\
	\texttt{HKLM\textbackslash SOFTWARE} & SOFTWARE\\
	\texttt{HLKM\textbackslash SYSTEM} & SYSTEM\\
\end{tabular}
Schlüssel \texttt{HKLM\textbackslash HARDWARE} mit dynamischen Hive, wird beim Systemstart erstellt aber nicht gespeichert\\
Liste zu Standard-Hive-Files: \texttt{HKLM\textbackslash SYSTEM\textbackslash CurrentControlSet\textbackslash Control\textbackslash hivelist}\\
Liste User-Hives: \texttt{HLKM\textbackslash SOFTWARE\textbackslash Microsoft\textbackslash Windows NT\textbackslash CurrentVersion\textbackslash ProfileList}

\section{SID \& SAM}
Liste der SIDs \texttt{HKLM\textbackslash Software\textbackslash Microsoft\textbackslash WindowsNT\textbackslash CurrentVersion\textbackslash ProfileList}\\
Pfad zu individuellen Profilen: \texttt{ProfileImagePath}\\
Aufbau der SID (S-1-5-21-[...]-1002):
\settowidth{\MyLen}{\texttt{SID.length.}}
\begin{tabular}{@{}p{\the\MyLen}%
		@{}p{\linewidth-\the\MyLen}@{}}
	\texttt{S} & identifiziert den Schlüssel als SID\\
	\texttt{1} & Revisionsnummer, Nummer der SID-Spezifikation\\
	\texttt{5} & Autorität\\
	\texttt{21-[...]} & Domänen-ID, identifiziert die Domäne oder den lokalen Computer, Wert ist variabel\\
	\texttt{1002} & Benutzer-ID, relative ID (RID), >1000 für Profile die nicht standardmäßig generiert wurden
\end{tabular}

\section{Wichtige Pfade}
\subsubsection{Autorun}
\settowidth{\MyLen}{\texttt{HKLM.Software.Microsoft.Windows.CurrentVersion.RunOnce.space}}
\begin{tabular}{@{}p{\the\MyLen}%
		@{}p{\linewidth-\the\MyLen}@{}}
	\texttt{HKLM\textbackslash Software\textbackslash Microsoft\textbackslash Windows\textbackslash CurrentVersion\textbackslash RunOnce} & \\
	\texttt{HKLM\textbackslash Software\textbackslash Microsoft\textbackslash Windows\textbackslash CurrentVersion\textbackslash Run} & \\
	\texttt{HKCU\textbackslash Software\textbackslash Microsoft\textbackslash Windows\textbackslash CurrentVersion\textbackslash RunOnce} & \\
	\texttt{HKCU\textbackslash Software\textbackslash Microsoft\textbackslash Windows\textbackslash CurrentVersion\textbackslash Run} & \\
\end{tabular}
Pfade in \texttt{Run} bei jedem Systemstart, \texttt{RunOnce} nur einmal

\subsubsection{MRU}
\texttt{\path{HKU\<SID>\Software\Microsoft\Windows\CurrentVersion\Explorer}}
\settowidth{\MyLen}{\texttt{RecentDocs.sp.}}
\begin{tabular}{@{}p{\the\MyLen}%
		@{}p{\linewidth-\the\MyLen}@{}}
	\texttt{ComDlg32} & Zuletzt ausgeführte Anwendungen und deren Pfade sowie geöffnete oder geänderte Dateien\\
	 \texttt{RecentDocs} & Unterschlüssel mit Dateierweiterungen, zuletzt geöffnete Dateien diesen Typs\\
	 \texttt{RunMRU} & Aufrufe, die via Run durchgeführt wurden\\
	 \texttt{UserAssist} & Werte von Objekten, auf der Nutzer zugegriffen hat (z.B. Optionen der Systemsteuerung, Dateiverknüpfungen und Programme)
\end{tabular}
ROT13 verschlüsselt, es gibt mehrere MRU-Listen in unterschiedlichen Listen

\subsubsection{Geschützter Speicher}
\texttt{\path{HKU\<SID>\Software\Microsoft\Protected Storage System Provider}}\\
Verschlüsselte Passwörter für viele Anwendungen (Outlook Express, MSN-Explorer oder Internet Explorer)\\
Autovervollständigung oder Passwort merken\\

\subsubsection{Internet Explorer}
\texttt{\path{HKU\<SID>\Software\Microsoft\Internet Explorer}}
\begin{tabular}{@{}p{\the\MyLen}%
		@{}p{\linewidth-\the\MyLen}@{}}
	\texttt{Download} & Informationen zu Downloads\\
	\texttt{Main} & Benutzereinstellungen (Search Bars, Startseite, etc.)\\
	\texttt{TypedURLs} & Zuletzt besuchte Seiten (z.B. EMail, Onlinebanking)\\
\end{tabular}
Microsoft Edge nutzt \texttt{\path{HKCU/Software/Classes/Local  Settings/Software/Microsoft/Windows/CurrentVersion/AppContainer/Storage/microsoft.microsoftedge_xxxxxx/MicrosoftEdge}}

\subsection{Netzwerke}
\subsubsection{WLAN}
\settowidth{\MyLen}{\texttt{HKLM.Software.Microsoft.Windows.spa.}}
\begin{tabular}{@{}p{\the\MyLen}%
		@{}p{\linewidth-\the\MyLen}@{}}
	\texttt{\path{HKLM/Software/Microsoft/Windows NT/CurrentVersions/NetworkCards}} & Netzwerkgeräte (Beschreibung und GUID)\\
	\texttt{\path{HKLM/System/CurrentControlSet/Services/Tcpip/Parameters/Interfaces/<GUID>}} & Details zum Netzwerkgerät (IP, Gateway, Domain)\\
\end{tabular}
\subsubsection{P2P}
\settowidth{\MyLen}{\texttt{HKLM.Software.Microsoft.Windows.sp.}}
\begin{tabular}{@{}p{\the\MyLen}%
		@{}p{\linewidth-\the\MyLen}@{}}
	\texttt{\path{HKLM/System/ControlSet001/Services/SharedAccess/Parameters/FirewallPolicy/StandardProfile/AuthorizedApplications/List}} & Applikationen mit erlaubtem Zugriff auf ausgehende Verbindungen\\
\end{tabular}

\subsection{Angeschlossene Geräte}
\settowidth{\MyLen}{\texttt{HKLM.Software.Microsoft.Windows.sp.}}
\begin{tabular}{@{}p{\the\MyLen}%
		@{}p{\linewidth-\the\MyLen}@{}}
	\texttt{\path{HKLM/System/Mounted Devices}} & Liste aller Geräte, die im System gemountet wurden\\
	\texttt{\path{HKCU/Software/Microsoft/Windows/CurrentVersion/Explorer/MountPoints2}} & Mount eines Geräts bei Nutzerlogin\\
	\texttt{\path{HKLM/System/CurrentControlSet/Control/DeviceClasses}} & Enthält für jede DeviceClass-GUID Unterschlüssel mit Geräten die verbunden waren oder sind. \texttt{DeviceInstance} ist Pfad zu \texttt{\path{HKLM/System/CurrentControlSet/Enum}}. Durch Export Zeitstempel für ersten und letzten Zugriff\\
	\texttt{\path{HKLM/System/CurrentControlSet/Enum/<Enumerator>/<DeviceID>}} & Geräte im System mit Gerätebeschreibung und IDs\\
\end{tabular}

\section{Antiforensische Maßnahmen}
\settowidth{\MyLen}{\texttt{HKLM.Software.Microsoft.}}
\begin{tabular}{@{}p{\the\MyLen}%
		@{}p{\linewidth-\the\MyLen}@{}}
	\texttt{\path{Zeitstempel fälschen}} & Prüfsumme häufig nur auf Inhalt (Tool \url{http://www.petges.lu/home/download})\\
	\texttt{Pagefile.sys} & In \texttt{\path{HKLM/System/CurrentCOntrolSet/Control/Session Manager/Memory Management}} den Wert \texttt{ClearPagefileAtShutdown} auf 1 setzen\\
	\texttt{Zeitstempel vermeiden} & \texttt{\path{HKLM/System/CurrentControlSet/Contol/FileSystem}} Wert \texttt{NtfsDisableLastAccessUpdate} auf 1 setzen\\
	\texttt{Einträge löschen} & Verlauf IE oder zuletzt genutzte Dokumente\\
	\texttt{UserAssist abstellen} & \texttt{\path{HKU/Software/Microsoft/Windows/CurrentVersion/Explorer/UserAssist}} Wert \texttt{NoLog} vom Typ DWORD mit Wert 1 erstellen
\end{tabular}

\section{Tools}
\settowidth{\MyLen}{\texttt{Tools.Registry.spa.}}
\begin{tabular}{@{}p{\the\MyLen}%
		@{}p{\linewidth-\the\MyLen}@{}}
	\texttt{FTK-Imager} & Erstellung von Abbildern, Kopien der Hive-Files (Live) (Files $\rightarrow$ Obtain Protected Files)\\
	\texttt{Registry-Editor} & Importieren und Exportieren von Dateien, Struktur laden und entfernen, Verbinden mit der Registry eines Remotecomputers, Berechtiungen ändern, Registry durchsuchen\\
	\texttt{RegShot} & Änderungen in der Registry aufzeichnen (Erstellen eines ersten Abbildes und Vergleich mit einem zweiten)\\
	\texttt{Forensic Registry EDitor (fred)} & Untersuchung und Bearbeitung von HIVE-Dateien, vorgefertige Berichtsvorlagen\\
	\texttt{RegRipper} & Extrahieren von spezifischen Informationen, Automatisierung durch Plugins und Profile\\
	\texttt{DCODE} & Decodieren von Zeitstempeln (\url{https://www.dcode.fr/timestamp-converter})\\
\end{tabular}
Umwandeln der Zeitrstempel

% You can even have references
\rule{0.3\linewidth}{0.25pt}
\scriptsize
\bibliographystyle{abstract}
\bibliography{refFile}
\end{multicols}